\documentclass{dcbl/challenge}

\setdoctitle{Files}
\setdocauthor{Stephan Bökelmann}
\setdocemail{sboekelmann@ep1.rub.de}
\setdocinstitute{AG Physik der Hadronen und Kerne}


\begin{document}
Subroutines, called by a process, that take parameters and return values are called \textbf{pure functions}.
They can be viewed as \textbf{stateless} and without \textbf{side-effects}.
Any pure function can be substituted by their retrun value.
This is the fundamental concept of programming.
In todays language the term \textbf{functional programing} is used to describe this concept, in opposition to many other modern programming paradigms.
But not everything, that our computer can execute is a pure function.
Often times, we want to write to use external hardware, or communicate with another program.
These interactions are called side-effects, they are not pure, and not stateless.
To interact with other programs, we need to have shared memory with them, so that one program can write to it and the other can read from it.
Sharing memory directly can lead to many intrecate challenges.
Thus a more convenient model for system-interaction is needed.
Such a model was defined in the early 1980s by \textbf{Richard Stallman} and became known as \textbf{POSIX}.
Among other things, POSIX defines a concept called a \textbf{stream}.
A stream is a portal, that the operating-system provides to a process. 
A process can either read from or write to a stream.
To use a stream, we now need to find a way, to:
\begin{enumerate}
    \item get a specific stream from the OS
    \item store reference to a stream behind an identifier
    \item use a function that returns contents of a stream
    \item use a function that writes to a stream
\end{enumerate} 


\section*{Exercises}
\begin{aufgabe}
    The library \texttt{stdio.h} defines a type - just as we have already defined an own struct - called a \texttt{FILE}.
    A file denotes a stream connection. 
    Among other information, it contains an address to a buffer, that was offered by the operating-system. 
    Since our RAM is the only memory, we are allowed to write to, we need to ask the operating-system to provide us with a buffer, that we can write to.
    Thus it is necessary, to make a systemcall to the operating system, whose return value is the address of the buffer.
    This whole shebang is encapsulated in a standard library function called \texttt{fopen}.
    Therefore we want to call the function \texttt{fopen} with two parameters:
    \begin{enumerate}
        \item the name of the filepath - which mal also be a device! Remember, everything is a file.
        \item the mode, that we want to open the file in, which can be \texttt{"r"}, \texttt{"w"}, \texttt{"a"}. The permissions are checked by the operating-system. THe 
    \end{enumerate}
\end{aufgabe}
\begin{aufgabe}
    Write a program, that takes a number as a sole parameter.
    Inside of the program, open a file in the directory, that the program is executed in and call it the number that was given as a parameter.
    Use a loop to write every number from 1 to the number that was given as a parameter to the file.
\end{aufgabe}

\begin{aufgabe}
    Next, we want to read a given file. This program resembles a command line tool that you have used during the past days. Which one is it?
    
    Write a program that takes a file-path as a sole parameter.
    Inside the program, read the file line by line and print every line to the terminal.
    Bonus: Try this with different types of files. Does your code work with binary files, csv, ...?
\end{aufgabe}

\begin{aufgabe}
    Finally, we want to read from one file and write to another to cover all potential types of operations with files. Just like the task before, this functionality is implemented in a command line tool. Which one is it?
    
    Write a program that takes two file-paths as parameters.
    Inside the program, read the first file line by line and write every line to the second file. Check that the content of both files is identical.
\end{aufgabe}


\section*{Annotations}
\begin{enumerate}
    \item nothing to see here
\end{enumerate}

\end{document}
